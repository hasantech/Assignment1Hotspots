\documentclass[14pt, a4paper]{article}


\usepackage{geometry}
 \geometry{
 a4paper,
 total={170mm,257mm},
 left=20mm,
 top=10mm,
 }


	\begin{document}
		
				
		\title{ MAKERERE UNIVERSITY \\COLLEGE OF COMPUTING AND INFORMATION SCIENCES\\ Concept document of The Makerere Hotspots }

		
		\author{\begin{tabular}{ |p{6cm}|p{5cm}|p{4cm}|  }
\hline
NAME & REGISTRATION NO. &STUDENT NO. \\
\hline
LUKWAO HASSAN NASSIL & 16/U/6666/EVE &216005795 \\
\hline
\end{tabular}}

		\date {\today}

		\maketitle


			\section{Introduction}
According to the previous findings on WIFI (Wireless Fidelity Alliance) information, students of the great Makerere university have been having problems of distinguishing between which WIFI belongs the university (free of use) and which doesn’t (for payment). Students have been issues with getting the SSID(Social Security Identification) or name, keys(or confidential information and passwords), the location, the range coverage and  number of users of the university WIFI connections. Its is reported that students have been walking around the university streets with their WIFI signals on in order to track and automatically connect to any open WIFI connection(no password required) they come across. This has been a hectic condition to mostly the freshmen  students who have been finding cone men and end up losing their properties like phones and laptops to them which led to the brainstorm of this new idea which is called the Makerere Hotspots.

				

			\section{Keywords}
			WI-FI: Wireless Fidelity Alliance.
SSID: Social Security Identification.


				

			\section{Background to the problem}
According to the current situation of the great Makrere, there is no proper way of communication WIFI information to their continuing and freshmen students. Students pay for a good and reliable internet hence they are entitled to it as it is part of their tuition payment. Hence a birth to this idea is to solve the problem of manually searching for WIFI connections along with their passwords by electronically collecting the information (SSID, keys, GPS location, average number of users, indoor/outdoor status, payment or free, signal strength, along with some images). That can be used by implementers (web programmers and Mobile application developers).
 

				
			\section{problem statement}
The deliverables of this project are electronically collected data and which will later n=be processed into information  about WIFIs around the great Makrere. The electronic data collected will be centrally managed by the google AppEngine cloud and can thus a reliable source of information about all the WIFI connections around the great Makerere.
			
			
			
			\section{Aim and objectives}
			
				\subsection{Aim or General Objective}
- To create a vast database of WIFI information around the great Makerere that will be centrally managed.

		
		
				
				\subsection{specific objectives}
-	To save students the budden of locating WIFI connections and and searching for their passwords hence saving time.\\
-	To save the network administrators the budden of circulating information about WIFI names and confidential information when changed.\\
-	To accurately provide the location of all the WIFI networks around the great Makerere university.\\
-	To lay ground for all the implementers that would like to use this information on the WIFI connections.\\
-	To provide students with comparisons among the different WI-FI connections which may be represented by peak hours or maximum number of students connected.\\


			
			\section{Research scope}
			This research project will stop at electronically collecting WI-FI information and centrally locating it with googleAppEngine and stop by Aggregating the information by providing statistical views which will be represented as pie charts or bar graphs.


			
			\section{Research Significance}
-	This may be used as relevant data to the network administrator of the university.
-	This data/information may be used in the different applications (both Web and Mobile) to be developed concerning WI-FI connections around Makerere University.
-	This information may be used to track internet fraud i.e. outside people who may access internet connections of the great Makerere illegally.
			
			

			

				
	

	\end{document}